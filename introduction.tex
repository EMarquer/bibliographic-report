\chapter*{Introduction}
\addcontentsline{toc}{chapter}{Introduction}

Logs are an important part of any computer ecosystem today but, to understand and make future decision on these logs is an hard and important task these days.
Deep Learning has emerged as a key player to complete this motto due to its state of the art performance on various major tasks related to anomaly detection. Among those tasks, there is intrusion detection, denial of service (DoS) attack detection, hardware and software system failures detection, and malware detection. However, in recent years deep learning is often regarded as black box due to the need of understanding very hard mathematical complexity and algorithms, and the lack of direct and simple interpretability.

Our project centers on anomaly detection in large amounts of system logs, with very few occurrences of said anomalies in available data. The project thus concentrates on realizing a top notch predictive system, used to model the normal behavior of the logs. Deviations from that behavior can be considered anomalies.

Based on those observations, we try to explain in this work a major deep learning architecture, the DAN (Deep Averaging Network), which we are using in our project and is also a state of the art architecture for major Natural Language Processing (NLP) tasks.
Apart from this work we also shed some light on other state of the art technique like RNN (Recurrent Neural Network) using attention mechanism, and LSTM (Long and Short Term Memory) Networks.
We plan to demonstrate our model’s performance and to illustrate its interpretability using the Los Alamos National Laboratory (LANL) cyber-security dataset in the upcoming parts. We will present that dataset along with our main dataset of industrial data from the PAPUD project in the last part of the report.

